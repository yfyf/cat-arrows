\usepackage{url}
\usepackage{hyperref}
\usepackage{fancyvrb}
\usepackage{csquotes}
\usepackage{upquote}
\usepackage{../lib/diagrams}

\usepackage{amsmath}
\usepackage{amssymb}
\usepackage{amsthm}
\usepackage{tikz}
\usetikzlibrary{matrix}

\newcommand{\catx}[1]{\mathbb{#1}}
\newcommand{\catC}{\catx{C}}
\newcommand{\catD}{\catx{D}}

\newcommand{\opcat}[1]{{#1}^{\text{op}}}
\newcommand{\Sets}{\mathbf{Sets}}
\newcommand{\Cat}{\mathbf{Cat}}
\newcommand{\Hask}{\mathbf{Hask}}

%% @TODO: invent some kid of inline 'code' style for this
\newcommand{\firstArr}{\texttt{first}}

%% @TODO: pick a 'math' style for arrow operations
\newcommand{\arrM}{\text{arr}}
\newcommand{\firstM}{\text{first}}
%% >>>, <<< are \ggg and \lll

\newcommand{\tocong}{\overset{\sim}{\to}}

\DefineVerbatimEnvironment{code}{Verbatim}{fontsize=\small,commandchars=\\\{\},codes={\catcode`$=3\catcode`_=8}}

\hypersetup{
    colorlinks,
    linkcolor=black,
    citecolor=black,
    filecolor=black,
    urlcolor=black
}

\newcommand{\binoidaldiag}{%
        \begin{diagram}%
        A \otimes B & \rTo^{f \ltimes B} & A' \otimes B \\%
    \dTo^{A \rtimes g}&                  &  \dTo_{A' \rtimes g} \\%
        A \otimes B' & \rTo_{f \ltimes B'} & A' \otimes B'%
        \end{diagram}%
}
