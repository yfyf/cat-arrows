\section{Arrows in Haskell}

If all that computer programs do would be to operate on a fixed number of inputs
to produce output at most one time (that is, to never halt or terminate with
the result), pure language semantics would be sufficient. But real programs do
much more than that. They are interactive, that is, they query the environment
for input during runtime, and depend on this additional input for their runtime
behavior. They cause effects in their environment by leveraging built-in
capabilities of the machine they run on. They may run indefinitely, all the
while periodically producing output.

All of these types of computation are impossible to achieve with a pure
language, so any general purpose programming language needs to have notions of
computation to capture all of the above, and more.

In Haskell, the most obvious type of a computation with input of type \verb|a|
and output of type \verb|b| is the pure function type \verb|a -> b|, which is
built from types \verb|a| and \verb|b| by the \verb|(->)| type constructor. It
seems to make sense to adopt this notion of a computation being from something
to something\footnote{We could always make it from or to some uninteresting
fixed thing if we don't care about parameters or output.}, so we will generally
expect any type of computation to be a two-parametric type.

One of the powerful features of Haskell is the possibility of treating functions
as values, empowering programmers to build up complex functionality from simple
functions by assembling them using \emph{function combinators}. It seems
sensible, though not necessary, to expect equivalent compositionality for other
types of computation. The most basic kind of combinator for functions is
composition, and we will require an analogue. Our common expectations for types
of computations can be expressed in Haskell using a \emph{type class}.

\begin{code}
  class Computation a where
      (>>>) :: a b c -> a c d -> a b d
\end{code}

This in essence states that any type \verb|a| can be made a computation by
giving an implementation of the ``composition'' function, \verb|(>>>)|,
combining two computations \verb|a b c| and \verb|a c d| to a computation
\verb|a b d|.

Under these preconditions it seems clear that any pure function should always
fulfill the requirements for any kind of computation, so we require additionally
an embedding of pure functions into computation types, arriving at type class

\begin{code}
  class Computation a where
      pure   :: (b -> c) -> a b c
      (>>>)  :: a b c -> a c d -> a b d\textrm{.}
\end{code}

It should become clear from the act of choosing common requirements for
computations that this is not the only possible interface. And in fact,
historically, it was not the first approach implemented in Haskell. As sketched
in the introduction, the related but similar notion of monads was adopted from
category theory with requirements described by something akin to

\begin{code}
  class Monad m where
      return :: a -> m a
      fmap   :: (a -> b) -> m a -> m b
      (>>=)  :: m a -> (a -> m b) -> m b\textrm{,}
\end{code}

with \verb|return| describing how to turn values of type \verb|a| into
computations of type \verb|m a|, \verb|fmap| explaining how \verb|m| is an
endofunctor on the category of types, and \verb|(>>=)| a combinator for monads
and Kleisli arrows. With this model of computation, it is possible to model
computations with a specific return type, but without regard to specifically
\emph{structured} input of the computation.

While this is convenient in many cases, the method of producing computations of
type \verb|m b| from values of type \verb|a| by means of function application
dooms the combinator \verb|(>>=)| to execute computation \verb|m a| before being
able to actually construct a computation \verb|m b|. This sequencing means that
it is impossible for the \verb|(>>=)| combinator to analyse commonalities in
the input structure. Unfortunately, this property of the monad structure entails
that certain optimisations are out of bounds.

This is disappointing, as it limits the practicability of the monad as a general
interface to computation. By the time this detriment became clear, illustrated
in the work on LL(1) parsers by Swierstra and Duponcheel (\cite{swierstra}), the
benefits of such a general interface had become clear however. This incited John
Hughes's proposal of a notion of computation along the lines of above's
\verb|Computation|, which he called \emph{arrows}\footnote{Why this name? The
authors could not find an explanation in the literature, but as each kind of
\emph{arrow} in Hughes's sense describes the arrows---that is, morphisms---in
some category of data types, this seems like a likely reason.}.

The \verb|Arrow| typeclass proposed in~\cite{hughes-monad2arr} contains one
additional request atop \linebreak\verb|Computation|, as this is actually
\emph{too} general. We can observe that its applicability is seriously
diminished by the fact it gives no general way of combining the \emph{results}
of computations. A concise way of enabling this kind of operation is the
addition of the function \verb|first|:

\begin{code}
  class Arrow a where
      arr   :: (b -> c) -> a b c
      (>>>) :: a b c -> a c d -> a b d
      first :: a b c -> a (b, d) (c, d)
\end{code}

This interface allows the definition of a kind of non-commutative product with
signature,

\begin{code}
  (***) :: Arrow a => a b c -> a b' c' -> a (b, b') (c, c')\textrm{.}
\end{code}

In addition to the constraints expressed by the type class, the semantics of the
functions are subject to a set of laws, the \emph{arrow laws}, which need to be
verified by the author of each instance's implementation.

\begin{code}[numbers=left]
                (a >>> b) >>> c == a >>> (b >>> c)
                    arr (g . f) == arr f >>> arr g
                   arr id >>> a == a
                              a == a >>> arr id
            first a >>> arr fst == arr fst >>> a
     first a >>> arr (id *** f) == arr (id *** f) >>> first a
  first (first a) >>> arr assoc == arr assoc >>> first a
                  first (arr f) == arr (f *** id)
                first (a >>> b) == first a >>> first b
\end{code}

Note that \verb|(***)| as used in the laws is the product on \emph{pure}
functions. The function \verb|assoc| translates tuples \verb|((a,b),c)| to
\verb|(a,(b,c))|, and the functions \verb|fst| and \verb|snd| produce the first
and second position of a tuple, respectively.

The obvious instance of \verb|Arrow| is the pure function type \verb|(->)|. With
\verb|(.)| being the composition function for pure functions, functions of the
adequate types can be given by

\begin{code}
  instance Arrow (->) where
      arr f     = f
      (>>>) f g = (.) g f
      first f   = \symbol{92}(x, y) -> (f x, y)\textrm{,}
\end{code}

where the right-hand expression in the definition of \verb|first| is an
anonymous function defined on pairs \verb|(x, y)|.

% TODO Example of an actual NON-pure computation?
