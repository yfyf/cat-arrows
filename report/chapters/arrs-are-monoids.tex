\section{Arrows are monoids}

\subsection{Arrows are endo-profunctors}

%% Disclaimer: I allowed myself to freestyle here, so please read with
%% suspicion.

We now explore a rather simplified interpretation of arrows, mostly following
the one initially described by Heunen \& Jacobs~\cite{arr-like-mon}. Our spirit
here is to just assume that the naive categorical interpretations of arrow
definitions are correct and then see what structure this gives us.
Using this approach we end up with the fact that arrows can be seen as monoids
on the category of bifunctors $\opcat{\catC} \times \catC \to \catC$.
Moreover, we also obtain a (sloppy) bijection between Freyd categories and
the aforementioned monoidal structure induced by arrows.

Since arrows are first of all a computational notion in functional programming,
to begin, we can, as usual, assume our working domain is some cartesian closed
category $\catC$. In particular, we can consider $\catC$ to be the category
$\Hask$ of Haskell types and functions.

Having this domain in mind, we can try to provide a categorical interpretation
of a Haskell \verb|(Arrow a)| instance. First, observe that every type
constructor \verb|f :: * -> *| in Haskell induces a subcategory of $\Hask$
where all types (i.e.~objects) are of the form \verb|f a|.  Similarly,
the \verb|(Arrow a)| type constructor \verb|a| of kind \verb|* -> (* -> *)|
also induces a subcategory of function types.\footnote{%
For intuition, observe that e.g. the function type constructor
\texttt{(->)} is of kind \texttt{* -> (* -> *)}}
%% can't use \verb in footnote for mysterious reasons

With these observations, we can deduce that it is reasonable to consider
\verb|(Arrow a)| categorically as, firstly, a map $A$, whose object part is
of the form $|\catC| \to |\catC \Rightarrow \catC|$, which after uncurrying
becomes $|A|: |\catC| \times |\catC| \to |\catC|$. Moreover, since we want $A$
to play nice with Haskell functions, we assume it is well-behaved with respect
to morphisms in $\Hask$, which boils down to $A$ being bifunctorial.

Moreover, we see $A$ is contravariant in the first variable and covariant in
the second. Thus we can conclude that $A$ is in fact an endo-profunctor on
$\catC$, i.e.~$A: \opcat{\catC} \times \catC \to \catC$. We later prove our
assumptions are correct\footnote{By which we mean: ``correct for the naive
interpretation''} and now consider the additional structure induced by
the arrow operations:

\begin{itemize}
    \item \verb|arr| is a function of type \verb|(b -> c) -> a b c|.
        The categorical interpretation of a function type is simply an internal
        hom in $\catC$, hence we get that categorically \verb|arr| corresponds
        to a functor sending homs in $\catC$ to homs in the subcategory induced
        by $A$, i.e.: $arr: [B \Rightarrow C] \mapsto A(B, C)$. Alternatively, one
        can say that $arr$ is a collection of natural transformations
        $arr_{BC}$, sending morphisms $B \to C$ in $\catC$ to morphisms in $A$;
    \item similarly, \verb|first| is a function of type
        \verb|a b c -> a (b, d) (c, d)|
        corresponding to a family of morphisms\\
        $first_{BCD}: A(B, C) \to A(B \times D,C \times D)$\\
        or as a functor $first$ sending homs $A(-, +)$ to homs $A(- \times D, +
        \times D)$;
    \item finally, \verb|(>>>)| is function of type \verb|a b c -> a c d -> a b d|.
        If $A$ were closed then we could interpret \verb|(>>>)| as a
        map\\
        $\ggg_{BCD}: A(B, C) \to (A(C, D) \Rightarrow A(B, D))$.\\
        However, this is not the case in general, so instead we can view
        \verb|(>>>)| as a family of morphisms\\
        $\ggg_{BCD}: A(B, C) \times A(C, D) \to A(B, D)$\\
\end{itemize}

We assume for now that our interpretations are correct (in particular, that the
claimed natural transformations are actually natural) and move on to take a
closer look at $\ggg$.

\begin{lemma}[{\cite[Lemma~3.3]{arr-like-mon}}]
    The maps $\ggg: A(X, P) \times (P, Y) \to A(X, Y)$ are natural in $X$,
    $Y$ and dinatural in $P$.
\end{lemma}

Advanced category theory wizards, upon spotting this dinaturality, immediately
realise that $\ggg$ can be used to obtain proper profunctor composition, which
also acts as a tensor product~\cite[p.~8]{arr-like-mon}. Hence we shall try to
follow their path.

\subsection{Arrows are monoids}

We now wish to discover a monoidal structure somewhere. The hardest part is
obviously recovering the hidden tensor product.

Since the only operation at hand which acts on some kind of products is $\ggg$
we concentrate our efforts on squeezing a tensor product out of it. We wish
to end up in a situation where $\ggg$ is a map
    \[ A \otimes A \overset{\ggg}{\longrightarrow} A \]

To do so, we first of all need to figure out what would an object of $A \otimes A$
look like. We do this by keeping in mind that we will want to map it into $A$
by using $\ggg$. Recall that $\ggg$ is actually a collection of morphisms
$\ggg_{BCD}$. This is somewhat inconvenient because we will want to have a
morphism of the form:
    \[ (A \otimes A)(X, Y) \overset{\ggg}{\longrightarrow} A(X, Y) \]
However, $\ggg_{BCD}$ includes a third ``middle'' object $C$.

This is where dinaturality comes in handy. Since dinaturality tells us that the
middle object $C$ is irrelevant, we can try to combine the domains of
$\ggg_{B\,\_ \,D}$ (ranging over the middle object) by lumping them up into one big
object.

The two usual choices for lumping things up are taking products and taking
coproducts. In this case, the right choice is clearly a
coproduct.\footnote{Because we are \emph{already} working with products, so
producing more products would not make any sense!}

Ignoring the apparent size issues, we can now consider $\ggg$ as
\[ \coprod_{C \in \catC} (A(B, C) \times A(C, D))
    \overset{\ggg}{\longrightarrow} A(B, D) \]

Is this huge coproduct already the thing we need? Not really, because in order
for it to play nice with the tensor product we need to also make sure the tensor
product is happy not only about that middle \emph{object} is ignored, but
also that this is done ``naturally'', i.e.~if we consider a morphism in
$\catC$, $f: C \to C'$, the tensor product should agree on the objects in the
presence of a morphism:
%% this makes no sense, but I want to go to sleep:
$\left( A(B, C) \times -\right) \mapsto
    \left( A(B, C') \times - \right)$

This boils down to~\cite[p.~10]{arr-like-mon} to coequalising our initial
coproduct to ensure the tensor object is what we need:

\[
\begin{diagram}
    \coprod_{C, C' \in \catC}
        (A(B, C) \times [C \Rightarrow C'] \times A(C', D)) &
        \pile{\rTo^{d_1} \\ \rTo_{d_2} } &
        \coprod_{C \in \catC} (A(B, C) \times A(C, D)) &
        \rTo^{c} &
        (A \otimes A)(B, D)\\
        & & & \rdTo & \dDashto_{\ggg} \\
        & & &       & A(B, D)
\end{diagram}
\]

as needed. $\bigstar$
