\section{Arrows are monoids}

\subsection{Arrows are endo-profunctors}

%% Disclaimer: I allowed myself to freestyle here, so please read with
%% suspicion.

We now explore a rather simplified interpretation of arrows, mostly following
the one initially described by Heunen \& Jacobs~\cite{arr-like-mon}. Our spirit
here is to just assume that the naive categorical interpretations of arrow
definitions are correct and then see what structure this gives us.
Using this approach we end up with the fact that arrows can be seen as monoids
on the category of bifunctors $\opcat{\catC} \times \catC \to \catC$.
Moreover, we also obtain a (sloppy) bijection between Freyd categories and
the aforementioned monoidal structure induced by arrows.

Since arrows are first of all a computational notion in functional programming,
to begin, we can, as usual, assume our working domain is some cartesian closed
category $\catC$. In particular, we can consider $\catC$ to be the category
$\Hask$ of Haskell types and functions.

Having this domain in mind, we can try to provide a categorical interpretation
of a Haskell \verb|(Arrow a)| instance. First, observe that every type
constructor \verb|f :: * -> *| in Haskell induces a subcategory of $\Hask$
where all types (i.e.~objects) are of the form \verb|f a|.  Similarly,
the \verb|(Arrow a)| type constructor \verb|a| of kind \verb|* -> (* -> *)|
also induces a subcategory of function types.\footnote{%
For intuition, observe that e.g. the function type constructor
\texttt{(->)} is of kind \texttt{* -> (* -> *)}}
%% can't use \verb in footnote for mysterious reasons

With these observations, we can deduce that it is reasonable to consider
\verb|(Arrow a)| categorically as, firstly, a map $A$, whose object part is
of the form $|\catC| \to |\catC \Rightarrow \catC|$, which after uncurrying
becomes $|A|: |\catC| \times |\catC| \to |\catC|$. Moreover, since we want $A$
to play nice with Haskell functions, we assume it is well-behaved with respect
to morphisms in $\Hask$, which boils down to $A$ being bifunctorial.

Moreover, we see $A$ is contravariant in the first variable and covariant in
the second. Thus we can conclude that $A$ is in fact an endo-profunctor on
$\catC$, i.e.~$A: \opcat{\catC} \times \catC \to \catC$. We later prove our
assumptions are correct\footnote{By which we mean: ``correct for the naive
interpretation''} and now consider the additional structure induced by
the arrow operations:

\begin{itemize}
    \item \verb|arr| is a function of type \verb|(b -> c) -> a b c|.
        The categorical interpretation of a function type is simply an internal
        hom in $\catC$, hence we get that categorically \verb|arr| corresponds
        to a functor sending homs in $\catC$ to homs in the subcategory induced
        by $A$, i.e.: $arr: [B \Rightarrow C] \mapsto A(B, C)$. Alternatively, one
        can say that $arr$ is a collection of natural transformations
        $arr_{BC}$, sending morphisms $B \to C$ in $\catC$ to morphisms in $A$;
    \item similarly, \verb|first| is a function of type
        \verb|a b c -> a (b, d) (c, d)|
        corresponding to a family of morphisms\\
        $first_{BCD}: A(B, C) \to A(B \times D,C \times D)$\\
        or as a functor $first$ sending homs $A(-, +)$ to homs $A(- \times D, +
        \times D)$;
    \item finally, \verb|(>>>)| is function of type \verb|a b c -> a c d -> a b d|.
        If $A$ were closed then we could interpret \verb|(>>>)| as a
        map\\
        $\ggg_{BCD}: A(B, C) \to (A(C, D) \Rightarrow A(B, D))$.\\
        However, this is not the case in general, so instead we can view
        \verb|(>>>)| as a family of morphisms\\
        $\ggg_{BCD}: A(B, C) \times A(C, D) \to A(B, D)$\\
        and $\ggg$ as a bifunctor.
\end{itemize}

We assume for now that our interpretations are correct (in particular, that the
claimed natural transformations are actually natural) and move on to take a
closer look at $\ggg$.

\begin{lemma}[{\cite[Lemma~3.3]{arr-like-mon}}]
    The maps $\ggg: A(X, P) \times A(P, Y) \to A(X, Y)$ are natural in $X$,
    $Y$ and dinatural in $P$.
\end{lemma}
\begin{proof}
Dinaturality means that for every map $f: P \to Q$, the following diagram
commutes:
\[ \begin{diagram}
    & & A(X, P) \times A(P, Y) & & \\
    & \ruTo^{\text{id} \times A(f, \text{id})} & & \rdTo^{\ggg_{XPY}} & \\
    A(X, P) \times A(Q, Y) & & & & A(X, Y) \\
   & \rdTo^{A(\text{id}, f) \times \text{id}} & & \ruTo^{\ggg_{XQY}} & \\
    & & A(X, Q) \times A(Q, Y) & &
\end{diagram} \]

Naturality is $X$ and $Y$ is easy to see as it is just pre- and post-
composition, which follows from bifunctoriality of $A$.

\newcommand{\idop}{\text{id}}

For dinaturality, assume $a: A(X, P)$ and $b: A(Q ,Y)$, then\footnote{Here $;$
is precomposition.}
\begin{align*}
    (\idop \times A(f, \idop);\ggg_{XPY})(a, b) =& a \ggg_{XPY} A(f, id)(b) \\
    =& a \ggg_{XPY} (arr(f) \ggg_{PQY} b) \\
    =& (a \ggg_{XPQ} arr(f)) \ggg_{XQY} b \\
    =& A(\idop, f)(a) \ggg_{XQY} b \\
    =& (A(\idop, f) \times \idop;\ggg_{XQY})(a,b)
\end{align*}
\end{proof}

Advanced category theory wizards, upon spotting this dinaturality, realise that
$\ggg$ can be used to obtain proper profunctor composition, which also acts as
a tensor product~\cite[p.~8]{arr-like-mon}. An elegant way to proceed then is
to use coends over the dinatural argument to extend the arrow precomposition
$\ggg$ to full profunctor composition and use it as a tensor object. This is
mentioned in \cite{cat-semantics-arr} and described in detail by
Asada~\cite{asada}. However, to avoid introducing extra concepts and keep our
presentation simple, we shall follow the construction by Jacobs
et.~al~\cite{cat-semantics-arr}.

\subsection{Arrows are monoids}

We now wish to discover some monoidal structure on the category of bifunctors
\\$\opcat{\catC} \times \catC \to \catC$. For that we need to recover the hidden
tensor product.

Since the only binary operation at hand is $\ggg$, we concentrate our efforts
on squeezing a tensor product out of it. We wish to end up in a situation where
$\ggg$ is a map \[ A \otimes A \overset{\ggg}{\longrightarrow} A. \]

To do so, we first of all need to figure out what would an object of $A \otimes
A$ look like. Recall that $\ggg$ is actually a collection of morphisms
$\ggg_{BCD}$. This is somewhat inconvenient because we will want to have a
morphism of the form: \[ (A \otimes A)(X, Y) \overset{\ggg}{\longrightarrow}
A(X, Y) \] However, $\ggg_{BCD}$ includes a third ``middle'' object $C$.

This is where dinaturality comes in handy. Since dinaturality tells us that the
middle object $C$ is irrelevant for composition, we can try to combine the domains of
$\ggg_{B\,\_ \,D}$ (ranging over the middle object) by lumping them up into one big
object.

The two usual choices for lumping things up are taking products and taking
coproducts. In this case, the right choice is clearly a
coproduct.\footnote{Because we are \emph{already} working with products, so
producing more products would not make any sense!}

Ignoring the apparent size issues, we can now consider $\ggg$ as
\begin{align}\label{ggg-coprod}
\coprod_{C \in \catC} (A(B, C) \times A(C, D))
    \overset{\ggg}{\longrightarrow} A(B, D)
\end{align}

Is this huge coproduct already the thing we need? Not yet, we also need to make
sure that, given $f: C \to C'$, the relation
\begin{equation}\label{eq:rel}
    A(-, f) \ggg A(C', -) = A(-, C) \ggg A(f, -)
\end{equation}
is preserved, i.e.~the dinaturality is captured also for morphisms.

This boils down to coequalising our initial coproduct to ensure the tensor
object preserves this property:

\begin{align}
    \begin{diagram}\label{megadiag}
    \coprod_{C, C' \in \catC}
        (A(B, C) \times [C \Rightarrow C'] \times A(C', D)) &
        \pile{\rTo^{d_1} \\ \rTo_{d_2} } &
        \coprod_{C \in \catC} (A(B, C) \times A(C, D)) &
        \rTo^{c} &
        (A \otimes A)(B, D)\\
        & & & \rdTo & \dDashto_{\ggg} \\
        & & &       & A(B, D)
\end{diagram}
\end{align}

where $d_1$, $d_2$ is are actually a collection of maps (for each $[C
\Rightarrow C']$), which ensure the relation~\eqref{eq:rel} is preserved via
\[ \begin{diagram}
    d_1: A(B, C) \times [C \Rightarrow C'] & \rTo^{\text{str}} &
    A(B, C \times [C \Rightarrow C']) & \rTo^{A(\text{id}, \text{eval})} & A(B, C')\\
    d_2: [C \Rightarrow C'] \times A(C', D) & \rTo^{\text{costr}} &
    A([[C \Rightarrow C'] \Rightarrow C'], D) & \rTo^{A(\text{eval}', \text{id})} &
    A(C, D)
\end{diagram} \]

Then $\ggg$ reappears by construction, because by the dinaturality condition it
clearly coequalizes the (collection of) maps $d_1, d_2$ and we have a
collection of maps $\ggg_{BCD}$ which then form the unique morphisms which
satisfy the universal property of the coequaliser.

Having dealt with the hard bit, we now only need to find a suitable unit
object for the monoidal structure. This is obviously just the identity
bifunctor, i.e.~the hom-functor $\catC(-, +)$, which we shall denote as $I$.

To obtain the needed isomorphism $A \cong A \otimes I$, we at first construct a
morphism $\rho: A \to A\otimes I$ by, for every map $P \to Y$ injecting $A(X,
Y)$ into the coproduct
\[ \coprod_{P \in \catC} A(B, P) \times [P\Rightarrow Y] \]
and coequalising such maps to obtain
    \[ (A \otimes I)(X, Y) \]
It is then possible to construct the inverse $\rho^{-1}$ by considering a
specific instance of the diagram to~\ref{megadiag} for the unit $I$ and a
cotuple of maps \[
    A(X, P) \times [P \Rightarrow Y] \to
    A(X, P \times [P \Rightarrow Y]) \to
    A(X, Y)
\]
and taking the unique map $(A\otimes I)(X, Y) \to
A(X, Y)$ given by the universal property as the inverse.
