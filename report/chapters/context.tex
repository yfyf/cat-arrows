\chapter{Context}

In the summer of 1987 a group of computer scientists got high on the
possibility of operating on infinite data structures through the use of
\emph{lazy}\footnote{Technically, the magic of treatable infinite data
structures comes from the use of \emph{non-strict} semantics, in which
expressions are evaluated outside-in, and which therefore allow to extract
partial results from non-terminating computations. Lazy evaluation, in which
expressions are only evaluated as needed, is merely one way of implementing
this behaviour. As however it is currently the only mainstream implementation
technique and is often used synonymously in the literature, we will follow this
habit.} programming language semantics~\cite[p. 12-3]{hask-history}. The
occasion was a tutorial session on the functional programming language
\emph{Miranda}, one of the more mature exemplars of the many functional
languages that sprang to life in the 1970s and 1980s. In the hope of liberation
from the \emph{von Neumann} style\footnote{Following John Backus's call for
higher level programming languages in~\cite{backus}.}, researchers had begun
conceptual and practical work on languages in which functions as language
features were better aligned with their mathematical brethren. While all
functional programming languages share the first-class malleability of
functions and consider functions as mappings between values, lazy semantics
take this idea to the extreme.

Since expressions are only evaluated when their values are required for
further computation, it becomes difficult to determine when a given
expression might be evaluated. And in fact, in many cases, the values of
certain expressions are computationally never needed, even though their
execution is vital to a given program. The obvious example is output.
From a purely computational perspective it is irrelevant, whereas for
practical purposes a program without output is utterly useless.

What, for example, would be the expected result of the list expression

\begin{code}
  [print 'a', print 'b', print 'c']
\end{code}

in a language which only evaluates this to

\begin{code}
  (print 'a'):[print 'b', print 'c']
\end{code}

until forced to actually produce a value for one of the members?

It becomes clear that the result of such an expression in a lazy
language would be dependent on how the expression is used in the
program. But this is undesirable, as it would be preferable to easily
determine the sequencing of output in a direct way.

For simplicity reasons, lazy semantics therefore suggest the additional
adoption of \emph{pure} semantics~\cite[p. 12-8]{hask-history}, that is, to
fully align functions in the programming language with (partial) functions in
the mathematical sense.  Hence a function maps each value of its domain to a
value in its (lifted) codomain, and does nothing else.

While purity of the language makes it easy to reason about programs, it
introduces a lot of difficulty in interacting with the real world. Since
interaction with the environment is not predictable, that is, not
functional in the mathematical sense (at least from the closed world of
the program), real word interaction such as keyboard input or random
number generation can not be solved using functions, as their value
under constant input is constant.

The intuitive idea of opening up the closed world of the program, and
making the \emph{world} part of the input would solve the problem, but
would be very cumbersome. Formal variations of this strategy were
actually implemented in initial versions of Haskell, but didn't
contribute to an image of a \emph{serious} programming language.
Solutions were inflexible and required manual adaptation of effectful
functions.

Concurrently Eugenio Moggi was independently working on a related problem in
denotational semantics for the $\lambda$-calculus. He was trying to devise a
method for proving program equivalence for a general notion of computation, as
opposed to the overly simplified method of $\beta\eta$-conversion, which
reduces computation to total mappings from values to values. Moggi was
interested in capturing ``behaviours like non-termination, non-determinism or
side-effects, that can be exhibited by real programs''~\cite[p.
1]{moggi-89}.

For this he was proposing a categorical semantics for programming languages in
which each type as object $B$ would be separate from its type of corresponding
\emph{computations} as object $TB$. With $T$ being called a ``notion of
computation'', the goal was as follows~\cite[p. 2]{moggi-89}:

\begin{quote}
Rather than focus on specific notions of computations, we will identify the
general properties that the object $TB$ of computations must have. The basic
requirement is that programs should form a category, and the obvious choice for
it is the Kleisli category for a monad.
\end{quote}

This caught the attention of Philip Wadler, who adopted the idea for practical
use in Haskell~\cite[p. 12-23]{hask-history}. While it was initially
interesting for its uniform interface to different computational patterns and
not immediately adopted as a solution to the problem of input/output and
side-effects, by the time the Haskell 98 standard was finished it had been
established as the standard solution to those problems.

Interestingly, by then there had already been a computational pattern
appearing in the Haskell community that did not fit the monad structure.
In order to implement combinatorial parsers for context-free grammars
efficiently, a different representation was desirable and implemented by
Swierstra and Duponcheel (1996).

A later paper by John Hughes generalized this notion of computation into
an interface named \emph{arrows}, born out of practical considerations
of actual language use. Now trying to go the other direction from
practical motivation to theory, the goal became to find a categorical
equivalent of arrows as described in Haskell.

% FIXME Rephrase everything properly ...
% TODO Somehow we are discussing the aptness of the Freyd
% category/arrows equivalence. Atkey is noting that things are not quite
% as they appear to be. At the same time, even for monads things are not
% quite as they appear to be. It is not even completely clear how a
% categorical interpretation of Haskell should look like.
%
% It would be interesting to note this, maybe in closing. In the end:
% Does Atkey even give a more ``proper'' interpretation? Or does he
% maybe just decide to look at Hask differently and thus generates a
% slightly different result?
%
% Speaking with Marx: The mathematicians have only interpreted Hask
% in various ways: the point, however, is to harvest good ideas.
