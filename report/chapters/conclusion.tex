\chapter{Conclusion}

We have now seen arrows from the perspective of functional programming, and
were able to relate the notion we encountered there with an independent
categorical notion. How well do the two relate? In a sense, as we have seen,
very well, but at the same time, the nature of programming language
semantics in the real world is rarely as tidy and well-behaved as one could
hope. To be able to look at the category of Haskell types and pure
functions, one has to be generous with the language semantics, thereby
modelling in an only \emph{morally} faithful way.

Then again, as was briefly mentioned in section~\ref{sec:cat-arrows-hist} on
the history of categorical interpretations of arrows, the argument given
in~\cite{cat-semantics-arr} and followed in this report ignores the fact that
in $\Hask$ one is dealing not with hom-sets, but hom-types. This is not an
essential flaw, but as seen in~\cite{atkey-fix}, the construction becomes more
complicated when taking hom-types into consideration.

On the other hand, the situation with arrows is not so different from that
of monads' appearance in functional programming, where similar amounts of
generosity are required. This has not hindered a very fruitful
exchange\footnote{An exchange which may be a bit one-sided.} between
category theory and functional programming from growing ever more
pronounced. While early publications such as~\cite{cats-and-cp} show that
this exchange has been flourishing for some time already, the influx of
categorical concepts into practical programming seems to have increased in
recent years, with a growing number of libraries to support programming in
category theory informed ways.
