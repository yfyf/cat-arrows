\section{Arrows are Freyd categories}

Having gained some acquaintance with both Freyd categories as well as arrows and
their categorical interpretation, we now return to the \emph{folklore} claim
encountered in \ref{sec:cat-arrows-hist}. We will follow the exposition
in~\cite{cat-semantics-arr} to give a formal meaning to the claim. To do so, we
will first give a way of constructing categories from arrows such that they give
rise to Freyd categories. Second, we will complete the correspondence by showing
how Freyd categories give rise to arrows.

\begin{definition}
    For an arrow $A : \opcat{\catC} \times \catC \to \catC$ with operations
    $arr, \ggg$, and $first$, define $\catC_A$ by
    \begin{itemize}
        \item $|\catC_A| := |\catC|,$
        % TODO Is this what is meant at 2009, p. 420?
        % TODO We are adapting to \catC-valued A, hence result is hom-type.
        %      Implications?
        \item $\catC_A(X, Y) := A(X, Y),$
        % TODO Mark respective cats?
        \item for $X \in \catC_A$, $1_X := arr(id_X),$
        \item for $f: X \to Y, g: Y \to Z, g \circ f := f \ggg g.$
    \end{itemize}
\end{definition}

\begin{proposition}
    $J_A : \catC \to \catC_A, X \mapsto X, f \mapsto arr(f)$ is a functor.
\end{proposition}

\begin{proof}
    This follows from the definition of $\catC_A$ and the arrow laws, whereby
    $J_A(g \circ f) = arr(g \circ f) = arr(f) \ggg arr(g) = g \circ_{\catC_A}
    f$.
\end{proof}

We shall now show that this gives rise to a Freyd category.

\begin{lemma}
    For an arrow $A : \opcat{\catC} \times \catC \to \catC$, the category
    $\catC_A$, together with $\catC$ and $J_A : \catC \to \catC_A$ forms a
    Freyd category.
\end{lemma}

% The proof, the proof, the proof is on fire.
\begin{proof}
    To see that $\catC_A$ is symmetric premonoidal, take $I$ to be the initial
    object of $\catC_A$ and define $X \otimes Y := X \times Y$ -- through which
    the associator, unitors and commutator are given as the obvious operations
    on products -- extended to a functor by setting, for $f: X \to Y$, $f
    \ltimes Z := first_Z(f)$, and $Z \rtimes f := second_Z(f)$.
\end{proof}
