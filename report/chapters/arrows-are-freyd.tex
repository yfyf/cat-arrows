\section{Arrows are Freyd categories}

Having gained some acquaintance with both Freyd categories as well as arrows and
their categorical interpretation, we now return to the \emph{folklore} claim
encountered in~\ref{sec:cat-arrows-hist}. We will follow the exposition
in~\cite{cat-semantics-arr} to give a formal meaning to the claim. To do so, we
will first give a way of constructing categories from arrows such that they give
rise to Freyd categories. Second, we will complete the correspondence by showing
how Freyd categories give rise to arrows.

Before however getting into the meat of this section, we want to summarize the
results of this chapter in so far in

\begin{definition}[Arrows categorically]\label{def:catarr}
    Let $\catC$ be a locally small category with finite products. Define an
    \emph{arrow over $\catC$} to be a monoid in the category of profunctors
    $\opcat{\catC} \times \catC \to \catC$ possessing a natural transformation
    $first$, whose components $first_Z: A(X, Y) \to A(X \times Z, Y \times Z)$
    satisfy the arrow laws (5)--(9).
\end{definition}

Now for the new part.

\begin{definition}
    For an arrow $A : \opcat{\catC} \times \catC \to \catC$ with operations
    $arr, \ggg$, and $first$, define $\catC_A$ by
    \begin{itemize}
        \item $|\catC_A| := |\catC|,$
        \item $\catC_A(X, Y) := A(X, Y),$
        % TODO Mark respective cats?
        \item for $X \in \catC_A$, $1_X := arr(id_X),$
        \item for $f: X \to Y, g: Y \to Z, g \circ f := f \ggg g$.
    \end{itemize}
\end{definition}

\begin{proposition}
    $J_A : \catC \to \catC_A, X \mapsto X, f \mapsto arr(f)$ is a functor.
\end{proposition}

\begin{proof}
    This follows from the definition of $\catC_A$ and the arrow laws, whereby
    $J_A(g \circ f) = arr(g \circ f) = arr(f) \ggg arr(g) = g \circ_{\catC_A}
    f$.
\end{proof}

We shall now show that this gives rise to a Freyd category.

\begin{lemma}
    For an arrow $A : \opcat{\catC} \times \catC \to \catC$ over a category
    $\catC$ with finite products, the category $\catC_A$, together with $\catC$
    and $J_A : \catC \to \catC_A$ forms a Freyd category.
\end{lemma}

% The proof, the proof, the proof is on fire.
\begin{proof}
    To see that $\catC_A$ is symmetric premonoidal, take $I$ to be the initial
    object of $\catC_A$ and define $\alpha_{X, Y, Z}^{\catC_A} := arr(\alpha_{X,
    Y, Z}), \lambda^{\catC_A}_X := arr(\lambda_X), \rho^{\catC_A}_X :=
    arr(\rho_X),$ and $\gamma^{\catC_A}_{X, Y} := arr(\gamma_{X, Y})$. The
    operation $X \otimes Y := X \times Y$ can be extended to a functor by
    setting, for $f: X \to Y$, $f \ltimes Z := first_Z(f)$, and $Z \rtimes f :=
    second_Z(f)$.
    % TODO Does this suffice to show $\catC_A$ is premonoidal?

    Since $J_A$ is the identity on objects and $arr$ on morphisms, it only
    remains to show that it also preserves central maps. Let then $f: X \to X'$
    be some morphism from $\catC$ (hence it is central). We need to verify that
    $J_Af$ is central in $\catC_A$. Let for this $g : Y \to Y'$ be some
    arbitrary morphism from $\catC_A$, then we want
    \[
    \begin{diagram}
        X \otimes Y         & \rTo^{J_Af \ltimes Y}  & X' \otimes Y \\
        \dTo^{X \rtimes g}  &                        & \dTo_{X' \rtimes g} \\
        X \otimes Y'        & \rTo^{J_Af \ltimes Y'} & X' \otimes Y'
    \end{diagram}
    \]
    to commute, which is just
    % TODO Introduce second
    \[
    \begin{diagram}
        X \otimes Y         & \rTo^{first_Y(arr(f))}    & X' \otimes Y \\
        \dTo^{second_X(g)}  &                           & \dTo_{second_{X'}(g)} \\
        X \otimes Y'        & \rTo^{first_{Y'}(arr(f))} & X' \otimes Y'
    \end{diagram}
    \]
    and follows from the arrow laws and the definition of $second$ by:
    \begin{align*}
        first_Y(arr(f)) \ggg second_{X'}(g)
          &= arr(f \times id_Y) \ggg second_{X'}(g) \\
          &= arr(f \times id_Y) \ggg arr(\gamma_{X'Y}) \ggg first_{X'}(g) \ggg arr(\gamma_{Y'X'}) \\
          &= arr((f \times id_Y) \circ \gamma_{X'Y}) \ggg first_{X'}(g) \ggg arr(\gamma_{Y'X'}) \\
          &= arr(\gamma_{XY} \circ (id_Y \times f)) \ggg first_{X'}(g) \ggg arr(\gamma_{Y'X'}) \\
          &= arr(\gamma_{XY}) \ggg arr(id_Y \times f) \ggg first_{X'}(g) \ggg arr(\gamma_{Y'X'}) \\
          &= arr(\gamma_{XY}) \ggg first_{X}(g) \ggg arr(id_{Y'} \times f) \ggg arr(\gamma_{Y'X'}) \\
          &= arr(\gamma_{XY}) \ggg first_{X}(g) \ggg arr((id_{Y'} \times f) \circ \gamma_{Y'X'}) \\
          &= arr(\gamma_{XY}) \ggg first_{X}(g) \ggg arr(\gamma_{Y'X} \circ (f \times id_{Y'})) \\
          &= arr(\gamma_{XY}) \ggg first_{X}(g) \ggg arr(\gamma_{Y'X}) \ggg arr(f \times id_{Y'}) \\
          &= arr(\gamma_{XY}) \ggg first_{X}(g) \ggg arr(\gamma_{Y'X}) \ggg first_{Y'}(arr(f)) \\
          &= second_X(g) \ggg first_{Y'}(arr(f)).
    \end{align*}
    For the other diagram,
    \[
    \begin{diagram}
        Y \otimes X        & \rTo^{Y \rtimes J_Af}  & Y \otimes X' \\
        \dTo^{g \ltimes X} &                        & \dTo_{g \ltimes X'} \\
        Y' \otimes X       & \rTo^{Y' \rtimes J_Af} & Y' \otimes X',
    \end{diagram}
    \]
    that is,
    \[
    \begin{diagram}
        Y \otimes X       & \rTo^{second_Y(arr(f))}    & Y \otimes X' \\
        \dTo^{first_X(g)} &                            & \dTo_{first_{X'}(g)} \\
        Y' \otimes X      & \rTo^{second_{Y'}(arr(f))} & Y' \otimes X',
    \end{diagram}
    \]
    the procedure is similar.
\end{proof}

\begin{lemma}
    Every Freyd category of the form $\catD, \catC, J: \catC \to \catD$ induces
    an arrow.
\end{lemma}

\begin{proof}
    Define $A: \opcat{\catC} \times \catC \to \catC$ by setting $A(X, Y) :=
    \catD(X,Y)$. We first need to verify that $A$ forms a monoid in the category
    of $\catC$-enriched profunctors $\opcat{\catC} \times \catC \to \catC$ with
    appropriate unit and multiplication.

    Let for this $arr(f) := Jf$ and $a \ggg b := b \circ_\catD a$. A quick
    verification of the arrow laws (1)--(4) can be done by noting that (1) is
    inherited from the composition of $\catD$, (2) holds since $J$ is a functor
    on $\catC$, and the same is true for (3) and (4).

    $first$ can be given as
    \[
        first_Z(a) := (arr(\lambda p : X \times Z. (p, p))
          \ggg ((arr(\pi_1) \ggg a) \ltimes X \times Z))
          \ggg arr(id_Y \times \pi_2).
    \]
    The checking of the arrow laws (5) through (9) for this definition is intricate
    and omitted\footnote{The proof in~\cite{cat-semantics-arr} uses the notion
    of \emph{internal strength} in place of $first$, which we avoid for
    terminological simplicity, but whereby this verification is complicated.}.
%    \begin{align*}
%        & first_Z(a) \ggg arr(\pi_1) \\
%        & \quad = \big(arr(\lambda p : X \times Z. (p, p))
%          \ggg ((arr(\pi_1) \ggg a) \ltimes X \times Z)\big)
%          \ggg arr(id_Y \times \pi_2) \ggg arr(\pi_1) \\
%        & \quad = \big(arr(\lambda p : X \times Z. (p, p))
%          \ggg ((arr(\pi_1) \ggg a) \ltimes X \times Z)\big)
%          \ggg arr(\pi_1 \circ (id_Y \times \pi_2)) \\
%        & \quad = \big(arr(\lambda p : X \times Z. (p, p))
%          \ggg ((arr(\pi_1) \ggg a) \ltimes X \times Z)\big)
%          \ggg arr(\pi_1) \\
%        & \quad = \big(arr(\lambda p : X \times Z. (p, p))
%          \ggg ((arr(\pi_1) \ltimes X \times Z) \ggg (a \ltimes X \times Z))\big)
%          \ggg arr(\pi_1) \\
%        & \quad \overset{(1)}{=} \big(arr(\lambda p : X \times Z. (p, p))
%          \ggg (arr(\pi_1) \ltimes X \times Z)\big)
%          \ggg \big((a \ltimes X \times Z) \ggg arr(\pi_1)\big) \\
%        & \quad = \big(arr(\lambda p : X \times Z. (p, p))
%          \ggg (arr(\pi_1) \ltimes X \times Z)\big)
%          \ggg a \\
%        & \quad = \big(arr(\lambda p : X \times Z. (p, p))
%          \ggg (arr(\pi_1) \ltimes X \times Z)\big)
%          \ggg a \\
%    \end{align*}
%    \begin{align*}
%        & first_Z(arr(f)) \\
%        & \quad = \big(arr(\lambda p : X \times Z. (p, p))
%                    \ggg ((arr(\pi_1) \ggg arr(f)) \ltimes X \times Z)
%                  \big)
%                  \ggg arr(id_Y \times \pi_2) \\
%        & \quad = \big(arr(\lambda p : X \times Z. (p, p))
%                    \ggg (arr(f \circ \pi_1) \ltimes X \times Z)
%                  \big)
%                  \ggg arr(id_Y \times \pi_2)
%    \end{align*}
\end{proof}
The two lemmas together establish the main result of~\cite{cat-semantics-arr}.
\begin{theorem}[``Arrows are Freyd categories'']
    Given any locally small category $\catC$ with finite products, there is a
    one-to-one correspondence between arrows $A$ over $\catC$ and locally small
    Freyd categories $\catC \to \catD$.
\end{theorem}
