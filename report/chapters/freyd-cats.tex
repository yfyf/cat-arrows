\section{Freyd categories}

Freyd-categories were introduced by Power \& Robinson~\cite{pow-rob} as a type
of symmetric premonoidal category where the monoidal structure is given by
(the usual) \emph{product} operation. In the paper, this is merely a specific
case of the general notion of symmetric premonoidal categories, which are
proposed as generalisations of monoidal categories for modelling
non-commutative computational effects, such as non-determinism, exceptions and
concurrency.

The actual name, \emph{Freyd-}, is given in a later paper by Power \&
Thielecke~\cite{pow-thie} where Freyd-categories are examined as generalisations
of Cartesian closed categories and closed Freyd-categories are shown to be
models of Moggi's computational $\lambda$-calculus.

It is important to note that, in particular, Power \& Robinson showed that
Freyd-categories generalise strong monads, which were proposed by
Moggi~\cite{moggi-89} as categorical constructs for modelling effectful
computation. Hence the claimed computational-notion generalisations are
directly proved via category theory.

We shall now introduce the necessary notions to describe Freyd-categories and
some basic results about them.

\begin{definition}[Binoidal category]
    A binoidal category is a category $\catC$ equipped with:
    \begin{enumerate}
        \item for each $(A, B)$ in $|\catC| \times |\catC|$, an object $A \otimes B$
            in $|\catC|$;
        \item for each $A$, a functor $(A \rtimes -) : B \mapsto A \otimes B$
        \item for each $A$, a functor $(- \ltimes A) : B \mapsto B \otimes A$
    \end{enumerate}
\end{definition}

The conditions imply that, $A \rtimes B = A \otimes B = A \ltimes B$,
justifying the usage of the notation $A \otimes B$. Observe that this
operation is defined only on objects of $\catC$ so far.

A binoidal category is characterised by the following diagram
\binoidaldiag
where the composites do not necessarily commute, which means we do \emph{not}
always have a bifunctor $(- \otimes -): A\otimes B \mapsto A' \otimes B'$.

We can define maps which are ``nice'' with respect to $\otimes$, i.e.~those
that make the diagram above commute:

\begin{definition}[Central morphism]
    A morphism $f: A \to B$ in a binoidal category is \emph{central} if for
    every morphism $g: A' \to B'$ the composites of maps (of the form $h
    \ltimes C$ or $C \rtimes h$, where $C$ is either $A$ or $B$ and $h$ is
    either $f$ or $g$) from $A \otimes A'$ to $B \otimes B'$ and from $A'
    \otimes A$ to $B' \otimes B$ agree.
\end{definition}

A natural transformation is central if its components are central.

Having these, we can now define the generalisation of monoidal category.

\begin{definition}[Premonoidal category]
    A \emph{premonoidal category} is a binoidal category equipped with:
    \begin{enumerate}
        \item an \emph{identity object} $I$
        \item for all $A, B, C$, a natural central \emph{associator}
            isomorphism $\alpha_{A,B,C}\colon (A \otimes B) \otimes C \tocong A
            \otimes (B \otimes C)$
        \item for each object $A$, natural central isomorphisms: \emph{left
            unitor} $\lambda_A\colon A \otimes I \tocong A$ and \emph{right unitor}
            $\rho_A\colon I \otimes A \tocong A$
    \end{enumerate}

    such that the usual monoidal category coherence conditions are satisfied,
    i.e.~the pentagon law for $\alpha$ and triangle laws for $\alpha$,
    $\lambda$, $\rho$ hold.

\end{definition}

\begin{example}[Described in \cite{gen-comp-eff-models}]
    Let $\catC$ be a category with finite products and $S$ a specified object
    in $\catC$ (which we regard as \emph{state} or \emph{context}).

    Define $\catx{K}$ as a category such that $|\catx{K}| := |\catC|$ and
    $\catx{K}(X, Y) := \catC(S \times X, S \times Y)$, with composition in
    $\catx{K}$ determined by the composition in $\catC$.

    For any object $X$ of $\catC$, one has evident functors
    $X \otimes -: \catx{K} \to \catx{K}$ and
    $- \otimes X: \catx{K} \to \catx{K}$ extending the product in $\catC$.

    The maps are not bifunctorial (with respect to $\otimes$), hence do not
    yield a monoidal structure on $\catx{K}$. However, they do  yield a
    premonoidal structure on $\catx{K}$.
\end{example}

A \emph{strict premonoidal category} is a premonoidal category in which all the
isomorphisms described above are identities, the $\otimes$ operator is
associative on objects and moreover $I$ is really an identity for $\otimes$.

Note that a strict premonoidal category need not be a monoidal one.

A \emph{symmetric premonoidal category} is a premonoidal category equipped with
a central natural isomorphism $\gamma_{A,B} : A\otimes B \cong B\otimes A$, with
the usual coherence conditions of symmetric monoidal categories.

The \emph{centre} of a premonoidal category $\catC$ is a subcategory $\catC'$,
such that $|\catC| = |\catC'|$ and all the morphisms in $\catC'$ are central.

The centre of a premonoidal category is a monoidal
category~\cite[Prop.~3.1]{pow-rob}.
Hence, a monoidal category is simply a premonoidal category in which all
morphisms are central.

A \emph{strict} (resp. \emph{symmetric}) \emph{premonoidal functor} is a functor which
preserves the strict (resp. symmetric) premonoidal structure and sends central
maps to central maps.

We can now define Freyd categories.

\begin{definition}[Freyd category]
    A Freyd category consists of a category with finite products $\catC$
    and an identity-on-objects strict symmetric premonoidal functor
    \[ J: \catC \to \catx{K} \]
    (hence $\catx{K}$ is a symmetric premonoidal category).
\end{definition}

The definitions of premonoidal categories, and especially Freyd categories,
might seem a bit involved at first sight, however they arise as quite
reasonable generalisations of monoidal categories when one deals with
computational effects. It is hard to phrase this better than one of the
``discoverers'' of Freyd categories, John Power~\cite{gen-comp-eff-models}:

%% @TODO: shorten this and maybe come-up with own motivation / examples
%% if there's time.
\begin{quote}
    The notion of Freyd-category has emerged over the past 15 years as a subtle
    generalisation of the notion of category with finite products. It allows
    one to model environments in call-by-value programming languages containing
    computational effects, notably the $\lambda_c$-calculus, a variant of the
    call-by-value $\lambda$-calculus designed specifically to allow one to account
    for computational effects. Starting with the notion of category with finite
    products, one obtains the notion of a symmetric monoidal category by
    dropping insistence upon the existence of diagonals and projections: in
    such situations, one usually speaks of a tensor product rather than a
    product, corresponding to the relaxation from cartesian logic to linear
    logic. If one further drops the insistence upon bifunctoriality of the
    tensor product, one obtains the notion of a symmetric premonoidal category.
    This corresponds logically to keeping the terms of linear logic but putting
    fewer of them equal.  Just as one has cartesian closed categories and
    symmetric monoidal closed categories, one can speak of closedness for a
    symmetric premonoidal category too. Finally, if one reinstates the
    assumption of finite product structure but only on a specified subcategory
    of a putative symmetric premonoidal category, one has the notions of
    Freyd-category and closed Freyd-category [\ldots]
\end{quote}
