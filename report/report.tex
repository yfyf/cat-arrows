\documentclass[12pt,a4paper,titlepage]{scrreprt}
\usepackage[utf8]{inputenc}
\usepackage[english]{babel}

\usepackage[backend=biber,style=alphabetic]{biblatex}
\usepackage{url}
\usepackage{hyperref}
\usepackage{fancyvrb}
\usepackage{csquotes}
\usepackage{upquote}

\usepackage{amsmath}
\usepackage{amssymb}
\usepackage{amsthm}

\newcommand{\catx}[1]{\mathbb{#1}}
\newcommand{\catC}{\catx{C}}
\newcommand{\catD}{\catx{D}}

\newcommand{\opcat}[1]{{#1}^{\text{op}}}
\newcommand{\Sets}{\mathbf{Sets}}
\newcommand{\Cat}{\mathbf{Cat}}

%% @TODO: invent some kid of inline 'code' style for this
\newcommand{\firstArr}{\texttt{first}}

\newtheorem{definition}{Definition}

\DefineVerbatimEnvironment{code}{Verbatim}{fontsize=\small,commandchars=\\\{\},codes={\catcode`$=3\catcode`_=8}}

\hypersetup{
    colorlinks,
    linkcolor=black,
    citecolor=black,
    filecolor=black,
    urlcolor=black
}

\bibliography{literature}

\title{Cat-arrows}
\author{%
    Johannes Emerich
        (\href{mailto:Johannes@emerich.de}{johannes@emerich.de})\\
    Ignas Vyšniauskas
        (\href{mailto:i.vysniauskas@gmail.com}{i.vysniauskas@gmail.com})
}
\date{}

\begin{document}


\maketitle

\begin{abstract}
    This is a report on Arrows.
\end{abstract}

\chapter{Context}

In the summer of 1987 a group of computer scientists got high on the
possibility of operating on infinite data structures through the use of
\emph{lazy}\footnote{Technically, the magic of treatable infinite data
structures comes from the use of \emph{non-strict} semantics, in which
expressions are evaluated outside-in, and which therefore allow to extract
partial results from non-terminating computations. Lazy evaluation, in which
expressions are only evaluated as needed, is merely one way of implementing
this behaviour. As however it is currently the only mainstream implementation
technique and is often used synonymously in the literature, we will follow this
habit.} programming language semantics\smartcite[p. 12-3]{hask-history}. The
occasion was a tutorial session on the functional programming language
\emph{Miranda}, one of the more mature exemplars of the many functional
languages that sprang to life in the 1970s and 1980s. In the hope of liberation
from the \emph{von Neumann} style\smartcite{backus}, researchers had begun
conceptual and practical work on languages in which functions as language
features were better aligned with their mathematical brethren. While all
functional programming languages share the first-class malleability of
functions and consider functions as mappings between values, lazy semantics
take this idea to the extreme.

Since expressions are only evaluated when their values are required for
further computation, it becomes difficult to determine when a given
expression might be evaluated. And in fact, in many cases, the values of
certain expressions are computationally never needed, even though their
execution is vital to a given program. The obvious example is output.
From a purely computational perspective it is irrelevant, whereas for
practical purposes a program without output is utterly useless.

What, for example, would be the expected result of the list expression

\begin{code}
  [print 'a', print 'b', print 'c']
\end{code}

in a language which only evaluates this to

\begin{code}
  (print 'a'):[print 'b', print 'c']
\end{code}

until forced to actually produce a value for one of the members?

It becomes clear that the result of such an expression in a lazy
language would be dependent on how the expression is used in the
program. But this is undesirable, as it would be preferable to easily
determine the sequencing of output in a direct way.

For simplicity reasons, lazy semantics therefore suggest the additional
adoption of \emph{pure} semantics\smartcite[p. 12-8]{hask-history}, that is, to
fully align functions in the programming language with (partial) functions in
the mathematical sense.  Hence a function maps each value of its domain to a
value in its (lifted) codomain, and does nothing else.

While purity of the language makes it easy to reason about programs, it
introduces a lot of difficulty in interacting with the real world. Since
interaction with the environment is not predictable, that is, not
functional in the mathematical sense (at least from the closed world of
the program), real word interaction such as keyboard input or random
number generation can not be solved using functions, as their value
under constant input is constant.

The intuitive idea of opening up the closed world of the program, and
making the \emph{world} part of the input would solve the problem, but
would be very cumbersome. Formal variations of this strategy were
actually implemented in initial versions of Haskell, but didn't
contribute to an image of a \emph{serious} programming language.
Solutions were inflexible and required manual adaptation of effectful
functions.

Concurrently Eugenio Moggi was independently working on a related problem in
denotational semantics for the $\lambda$-calculus. He was trying to devise a
method for proving program equivalence for a general notion of computation, as
opposed to the overly simplified method of $\beta\eta$-conversion, which
reduces computation to total mappings from values to values. Moggi was
interested in capturing ``behaviours like non-termination, non-determinism or
side-effects, that can be exhibited by real programs''\smartcite[p.
1]{moggi-89}.

For this he was proposing a categorical semantics for programming languages in
which each type as object $B$ would be separate from its type of corresponding
\emph{computations} as object $TB$. With $T$ being called a ``notion of
computation'', the goal was as follows\smartcite[p. 2]{moggi-89}:

\begin{quote}
Rather than focus on specific notions of computations, we will identify the
general properties that the object $TB$ of computations must have. The basic
requirement is that programs should form a category, and the obvious choice for
it is the Kleisli category for a monad.
\end{quote}

This caught the attention of Philip Wadler, who adopted the idea for practical
use in Haskell\smartcite[p. 12-23]{hask-history}. While it was initially
interesting for its uniform interface to different computational patterns and
not immediately adopted as a solution to the problem of input/output and
side-effects, by the time the Haskell 98 standard was finished it had been
established as the standard solution to those problems.

Interestingly, by then there had already been a computational pattern
appearing in the Haskell community that did not fit the monad structure.
In order to implement combinatorial parsers for context-free grammars
efficiently, a different representation was desirable and implemented by
Swierstra and Duponcheel (1996).

A later paper by John Hughes generalized this notion of computation into
an interface named \emph{arrows}, born out of practical considerations
of actual language use. Now trying to go the other direction from
practical motivation to theory, the goal became to find a categorical
equivalent of arrows as described in Haskell.

% FIXME Rephrase everything properly ...
% TODO Somehow we are discussing the aptness of the Freyd
% category/arrows equivalence. Atkey is noting that things are not quite
% as they appear to be. At the same time, even for monads things are not
% quite as they appear to be. It is not even completely clear how a
% categorical interpretation of Haskell should look like.
%
% It would be interesting to note this, maybe in closing. In the end:
% Does Atkey even give a more ``proper'' interpretation? Or does he
% maybe just decide to look at Hask differently and thus generates a
% slightly different result?
%
% Speaking with Marx: The mathematicians have only interpreted Hask
% in various ways: the point, however, is to harvest good ideas.

\chapter{Modelling Computation in Theory and Practice}
    \section{Arrows in Haskell}

If all that computer programs do would be to operate on a fixed number of inputs
to produce output at most one time (that is, to never halt or terminate with
the result), pure language semantics would be sufficient. But real programs do
much more than that. They are interactive, that is, they query the environment
for input during runtime, and depend on this additional input for their runtime
behavior. They cause effects in their environment by leveraging built-in
capabilities of the machine they run on. They may run indefinitely, all the
while periodically producing output.

All of these types of computation are impossible to achieve with a pure
language, so any general purpose programming language needs to have notions of
computation to capture all of the above, and more.

In Haskell, the most obvious type of a computation with input of type \verb|A|
and output of type \verb|B| is the pure function type \verb|A -> B|, which is
built from types \verb|A| and \verb|B| by the \verb|(->)| type constructor. It
seems to make sense to adopt this notion of a computation being from something
to something\footnote{We could always make it from or to some uninteresting
fixed thing if we don't care about parameters or output.}, so we will generally
expect any type of computation to be a two-parametric type.

One of the powerful features of Haskell is the possibility of treating functions
as values, empowering programmers to build up complex functionality from simple
functions by assembling them using \emph{function combinators}. It seems
sensible, though not necessary, to expect equivalent compositionality for other
types of computation. The most basic kind of combinator for functions is
composition, and we will require an analogue. Our common expectations for types
of computations can be expressed in Haskell using a \emph{type class}.

\begin{code}
class Computation a where
    (>>>) :: a b c -> a c d -> a b d
\end{code}

This in essence states that any type \verb|a| can be made a computation by
giving an implementation of the ``composition'' function, \verb|(>>>)|,
combining two computations \verb|a b c| and \verb|a c d| to a computation
\verb|a b d|.

Under these preconditions it seems clear that any pure function should always
fulfill the requirements for any kind of computation, so we require additionally
an embedding of pure functions into computation types, arriving at type class

\begin{code}
class Computation a where
    arr   :: (b -> c) -> a b c
    (>>>) :: a b c -> a c d -> a b d
\end{code}

    \section{Freyd categories}

Freyd-categories were introduced by Power \& Robinson~\cite{pow-rob} as a type
of symmetric premonoidal category where the monoidal structure is given by
(the usual) \emph{product} operation. In the paper, this is merely a specific
case of the general notion of symmetric premonoidal categories, which are
proposed as generalisations of monoidal categories for modelling
non-commutative computational effects, such as non-determinism, exceptions and
concurrency.

The actual name, \emph{Freyd-}, is given in a later paper by Power \&
Thielecke~\cite{pow-thie} where Freyd-categories are examined as generalisation
of Cartesian closed categories and closed Freyd-categories are shown to be
models of Moggi's computational $\lambda$-calculus.

It is important to note that, in particular, Power \& Robinson showed that
Freyd-categories generalise strong monads, which were proposed by
Moggi~\cite{moggi-89} as categorical constructs for modelling effectful
computation. Hence the claimed computational-notion generalisations are
directly proved via category theory.

We shall now introduce the necessary notions to describe Freyd-categories and
some basic results about them.

\begin{definition}[Binoidal category]
    A binoidal category is a category $\catC$ equipped with:
    \begin{enumerate}
        \item for each $(A, B)$ in $|\catC| \times |\catC|$, an object $A \otimes B$
            in $|\catC|$;
        \item for each $A$, a functor $(A \rtimes -) : B \mapsto A \otimes B$
        \item for each $A$, a functor $(- \ltimes A) : B \mapsto B \otimes A$
    \end{enumerate}
\end{definition}

The conditions imply that, $A \rtimes B = A \otimes B = A \ltimes B$,
justifying the usage of the notation $A \otimes B$. Observe that this
operation is defined only on objects of $\catC$ so far.

\begin{definition}[Central morphism]
    A morphism $f: A \to B$ in a binoidal category is \emph{central} if for
    every morphism $g: A' \to B'$ the composites of maps (of the form $h
    \ltimes C$ or $C \rtimes h$, where $C$ is either $A$ or $B$ and $h$ is
    either $f$ or $g$) from $A \otimes A'$ to $B \otimes B'$ and from $A'
    \otimes A$ to $B' \otimes B$ agree.

    TODO: insert diagram
\end{definition}

A natural transformation is central if its components are central.

Having these, we can now define the generalisation of monoidal category.

\begin{definition}[Premonoidal category]
    A \emph{premonoidal category} is a binoidal category equipped with:
    \begin{enumerate}
        \item an \emph{identity object} $I$
        \item for all $A, B, C$, a natural central \emph{associator}
            isomorphism $\alpha_{A,B,C}\colon (A \otimes B) \otimes C \to A
            \otimes (B \otimes C)$
        \item for each object $A$, natural central isomorphisms: \emph{left
            unitor} $\lambda_A\colon A \otimes I \to A$ and \emph{right unitor}
            $\rho_A\colon I \otimes A \to A$
    \end{enumerate}

    such that the usual monoidal category coherence conditions are satisfied,
    i.e.~the pentagon law for $\alpha$ and triangle laws for $\alpha$,
    $\lambda$, $\rho$ hold.

\end{definition}

A \emph{strict premonoidal category} is a premonoidal category in which all the
isomorphisms described above are identities, the $\otimes$ operator is
associative on objects and moreover $I$ is really an identity for $\otimes$.

Note that a strict premonoidal category need not be a monoidal one.

A \emph{symmetric premonoidal category} is a premonoidal category equipped with
a central natural isomorphism $A\otimes B \cong B\otimes A$, with the usual
coherence conditions of symmetric monoidal categories.

The \emph{centre} of a premonoidal category $\catC$ is a subcategory $\catC'$,
such that $|\catC| = |\catC'|$ and all the morphisms in $\catC'$ are central.

The centre of a premonoidal category is a monoidal
category~\cite[Prop.~3.1]{pow-rob}.
Hence, a monoidal category is simply a premonoidal category in which all
morphisms are central.

We can now define Freyd categories.

\begin{definition}[Freyd category]
    A Freyd category consists of a category with finite products $\catC$
    and an identity-on-objects strict symmetric premonoidal functor
    \[ J: \catC \to \catx{K} \]
    (hence $\catx{K}$ is a symmetric premonoidal category).
\end{definition}

    \chapter{A short history of the categorical aspects of arrows}

Unlike Monads, which were borrowed from category theory to model various
computational behaviours, arrows were at first introduced by
Hughes~\cite{hughes-monad2arr} purely as a computational concept. Hence
categorical \emph{interpretations} of arrows came only as an afterthought.
Unsurprisingly, a whole family of interpretations was thus born, which we try
to describe briefly here. % and possibly expand on it later on

The tale of the categorical semantics of arrows begins with a folklore
statement
\begin{displayquote}Arrows are Freyd categories.\end{displayquote}
%% wtf -- this quoting is ugly -- want emph + quotemarks @TODO: fix
which is perhaps first noted by Paterson~\cite{paterson}.

This statement was first verified in detail by Heunen and
Jacobs~\cite{arr-like-mon}. Since computationally arrows are a generalisation
of monads, it is perhaps not so surprising that the situation turns out to be
similar categorically. Namely, Heunen and Jacobs prove that arrows can be
seen as monoids in categories of bifunctors $\opcat{\catC} \times \catC \to
\catC$, just like monads are monoids in a category of endofunctors.
Furthermore, they show there is a bijective correspondence between locally
small Freyd categories $\catC \to \catD$ and arrows over $\catC$.

However, the construction is slightly flawed due to issues with the size of
$\catC$.\footnote{$\catC$ would need to be both small and (co)complete, however
this is impossible~\cite[Chapter 3]{freyd-abelian-cats}} %% @TODO: insert citation
Hence, the statement \enquote{Arrows are Freyd} is only proved for bifunctors
on $\opcat{\catC} \times \catC \to \Sets$, where $\catC$ is \emph{small} (i.e.
profunctors) with an assumption that it would be possible to achieve the same
result in an enriched setting.

A later paper by Hasuo and Jacobs called ``Freyd is Kleisli, for
Arrows''~\cite{freyd-is-kleisli} elaborates on the fact that the correspondence
is actually an instance of the well-known Kleisli construction for monads,
adapted for arrows. This is again goes along with the general intuition that
arrows are a generalisation of monads.

%% @TODO: I'm not sure whether my interpretation of this is fully correct.
%% In particular, I need to check whether they are just examining the same
%% correspondance, or building something new.
%% They seem to be talking about a 'monoidal' interpretation and a Freyd one,
%% as of two different things...
%% Perhaps it's best to re-iterate this statement after we write-up the
%% 'meat' parts.

Another paper Jacobs, Heunen and Hasuo~\cite{cat-semantics-arr} combines the
previous results into a self-contained paper and will be our primary reference.

Finally, a paper by Atkey~\cite{atkey-fix} notes that the work of Jacobs et.~al
slightly misinterprets the situation. Firstly, Atkey elaborates on the fact
that it is necessary to consider enriched Freyd categories, because in order to
provide denotational semantics for arrows one needs to consider a \emph{type}
of morphisms between objects (which are types too), hence one needs at least
some sort of self-enrichment. More importantly, Atkey observes that the
\firstArr{} operator allows arrows to take \emph{unstructured} input. This does
not fit-in with the simplified ``Arrows are Freyd'' view, because it implies
that all computations are structured, which prevents modelling unstructured
input. Atkey thus shows that one also needs to consider indexed Freyd categories and
proves various relations between \{closed, indexed, enriched\} Freyd categories,
arrows and strong monads.

\chapter{Categorical semantics of arrows (?)}

\printbibliography

\end{document}
