\section{Freyd categories}

\begin{frame}
    \begin{center}\Huge Freyd categories\end{center}
\end{frame}

\begin{frame}

    {\huge \center Cartesian closed categories}
    \begin{itemize}
        \item CCC $\sim$ typed $\lambda$-calculus;
        \item intuitionistic logic $\sim$ typed $\lambda$-calculus;
        \item ($\Rightarrow$) intuitionistic logic $\sim$ CCC;
        \item also $\Hask$ is "morally" a CCC
    \end{itemize}
\end{frame}

\begin{frame}
    \begin{center}
    \huge
    Cartesian \sout{closed} categories

    $=$

    categories with finite products
    \end{center}

    this structure is characterised by

    \begin{itemeyez}
        \item[projections] $\pi_i: A_1 \times A_2 \to A_i$
        \item[symmetry] $A \times B \cong B \times A$
        \item[associativity] $A \times (B \times C) \cong (A \times B) \times C$
    \end{itemeyez}
\end{frame}


\begin{frame}
    \begin{center}\huge Symmetric monoidal categories\end{center}
    \begin{itemeyez}
        \item[\sout{projections}] \sout{$\pi_i: A_1 \times A_2 \to A_i$, $i = \{1, 2\}$}
        \item[symmetry] $A \otimes B \cong B \otimes A$
        \item[associativity] $A \otimes (B \otimes C) \cong (A \otimes B) \otimes C$
    \end{itemeyez}

\end{frame}

\begin{frame}
    \begin{center}\LARGE (Symmetric) {\bf pre}monoidal categories\end{center}
    \begin{itemize}
        \item Drop bifunctoriality of $(- \otimes -)$
        \item still a functor in each variable: $(- \otimes X)$, $(X \otimes -)$
        \item replace $\otimes$ with $\ltimes$ and $\rtimes$
        \item the following diagram does \emph{not} necessarily commute
        \binoidaldiag
        \item intuition: "external state"
    \end{itemize}
\end{frame}

\begin{frame}
\begin{example}[John Power]
    \begin{itemize}
        \item $\catC$ - Cartesian category (with $\times$)
        \item a specified object $S \in |\catC|$ ("state")
        \item Define $\catx{K}$ as
        \begin{itemize}
            \item $|\catx{K}| := |\catC|$
            \item $\catx{K}(X, Y) := \catC(S \times X, S \times Y)$
            \item $\circ_{\catx{K}} := \circ_{\catC}$
        \end{itemize}
    \end{itemize}
    Again we have functors:
    \begin{align*}
    X \otimes -: \catx{K} \to \catx{K} \\
    - \otimes X: \catx{K} \to \catx{K}
    \end{align*}
    They are \emph{not} bifunctorial!

    $\catx{K}$ is a symmetric premonoidal.
\end{example}
\end{frame}

\begin{frame}
\begin{example}[continued]
    Look at the diagram again
    \binoidaldiag
    Intuition:
    \begin{itemize}
        \item values computed by composites are the same
        \item side-effects on state are not
        \item consider $S = \{0, 1\}$, $f_S = (0)$, $g_S = (1)$
        \item upper composite results in state $1$, lower composite produces $0$.
    \end{itemize}
\end{example}
\end{frame}

\begin{frame}
\begin{definition}[Binoidal category]
    A binoidal category is a category $\catC$ equipped with:
    \begin{enumerate}
        \item for each $(A, B)$ in $|\catC| \times |\catC|$, an object $A \otimes B$
            in $|\catC|$;
        \item for each $A$, a functor $(A \rtimes -) : B \mapsto A \otimes B$
        \item for each $A$, a functor $(- \ltimes A) : B \mapsto B \otimes A$
    \end{enumerate}
\end{definition}

The conditions imply that, $A \rtimes B = A \otimes B = A \ltimes B$,
justifying the usage of the notation $A \otimes B$.
\end{frame}

\begin{frame}
\begin{definition}[Central morphism]
    A morphism $f: A \to B$ in a binoidal category is \emph{central} if for
    every morphism $g: A' \to B'$ the composites of maps (of the form $h
    \ltimes C$ or $C \rtimes h$, where $C$ is either $A$ or $B$ and $h$ is
    either $f$ or $g$) from $A \otimes A'$ to $B \otimes B'$ and from $A'
    \otimes A$ to $B' \otimes B$ agree.
\end{definition}
A natural transformation is central if its components are central.
\end{frame}


\begin{frame}
\begin{itemize}
    \item \emph{centre} of a premonoidal category $\catC$ is a subcategory
        $Z(\catC)$, such that $|Z(\catC)| = |\catC|$ and all the morphisms in
        $Z(\catC)$ are central.

    \item $Z(\catC)$ is monoidal

    \item ($\Rightarrow$) $\catC$ - monoidal $\iff$ $Z(\catC) = \catC$
\end{itemize}
\end{frame}

\begin{frame}
\begin{definition}[Premonoidal category]
    A \emph{premonoidal category} is a binoidal category equipped with:
    \begin{enumerate}
        \item identity $I$
        \item central $\alpha_{A,B,C}\colon (A \otimes B) \otimes C \tocong
            \otimes (B \otimes C)$
        \item central $\lambda_A\colon A \otimes I \tocong A$
        \item central $\rho_A\colon I \otimes A \tocong A$
    \end{enumerate}

    such that the usual monoidal category coherence conditions are satisfied,
    i.e.~the pentagon law for $\alpha$ and triangle laws for $\alpha$,
    $\lambda$, $\rho$ hold.
\end{definition}
\end{frame}

\begin{frame}
\begin{itemize}
    \item \emph{symmetric premonoidal category} is a premonoidal category
        equipped with a central natural isomorphism $\gamma_{A,B} : A\otimes B
        \cong B\otimes A$

    \item \emph{symmetric premonoidal} functor is a functor which preserves the
        symmetric premonoidal structure and sends central maps to central maps.
\end{itemize}
\end{frame}

\begin{frame}
\begin{definition}[Freyd category]
    A Freyd category consists of a category with finite products $\catC$
    and an identity-on-objects symmetric premonoidal functor
    \[ J: \catC \to \catx{K} \]
    (hence $\catx{K}$ is a symmetric premonoidal category).
\end{definition}
\end{frame}
