\section{Arrows are Freyd categories}

\frame{\tableofcontents[currentsection]}

\begin{frame}
\begin{definition}[Arrows categorically]
    Let $\catC$ be a locally small category with finite products. \\
    \vspace{1em}
    Define an
    \emph{arrow over $\catC$} to be a monoid in the category of profunctors
    $\opcat{\catC} \times \catC \to \catC$ possessing a natural transformation
    $first$, whose components $first_Z: A(X, Y) \to A(X \times Z, Y \times Z)$
    satisfy the arrow laws (5)--(9).
\end{definition}
\end{frame}

\begin{frame}
\begin{definition}
    For an arrow $A : \opcat{\catC} \times \catC \to \catC$ with operations
    $arr, \ggg$, and $first$, define $\catC_A$ by
    \begin{itemize}
        \item $|\catC_A| := |\catC|,$
        \item $\catC_A(X, Y) := A(X, Y),$
        \item for $X \in \catC_A$, $1_X := arr(id_X),$
        \item for $f: X \to Y, g: Y \to Z, g \circ f := f \ggg g$.
    \end{itemize}
\end{definition}

\begin{proposition}
    $J_A : \catC \to \catC_A, X \mapsto X, f \mapsto arr(f)$ is a functor.
\end{proposition}
\end{frame}

\begin{frame}
\begin{lemma}
    For an arrow $A : \opcat{\catC} \times \catC \to \catC$ over a category
    $\catC$ with finite products, the category $\catC_A$, together with $\catC$
    and $J_A : \catC \to \catC_A$ forms a Freyd category.
\end{lemma}
\end{frame}

\begin{frame}
\begin{proof}
\renewcommand{\qedsymbol}{}
    $\catC_A$ is symmetric premonoidal:
    \begin{itemize}
        \item Take $I$ to be the initial object of $\catC_A$,
        \item $\alpha_{X, Y, Z}^{\catC_A} := arr(\alpha_{X, Y, Z})$,
        \item $\lambda^{\catC_A}_X := arr(\lambda_X)$,
        \item $\rho^{\catC_A}_X := arr(\rho_X)$, and
        \item $\gamma^{\catC_A}_{X, Y} := arr(\gamma_{X, Y})$.
    \end{itemize}
    Extend $X \otimes Y := X \times Y$ to a functor, setting
    $f \ltimes Z := first_Z(f)$, and $Z \rtimes f := second_Z(f)$.
\end{proof}
\end{frame}

\begin{frame}
\begin{proof}[Proof (contd.)]
\renewcommand{\qedsymbol}{}
    $J_A$ preserves central maps, i.e. for $f \in \catC$, $J_Af$ is central in
    $\catC_A$. \\
    \vspace{1em}
    Thus for $g \in \catC_A$ the diagrams
    \begin{columns}[c]
    \begin{column}{.5\textwidth}
    \[
    \begin{diagram}
        X \otimes Y         & \rTo^{J_Af \ltimes Y}  & X' \otimes Y \\
        \dTo^{X \rtimes g}  &                        & \dTo_{X' \rtimes g} \\
        X \otimes Y'        & \rTo^{J_Af \ltimes Y'} & X' \otimes Y',
    \end{diagram}
    \]
    \end{column}
    \begin{column}{.5\textwidth}
    \[
    \begin{diagram}
        Y \otimes X        & \rTo^{Y \rtimes J_Af}  & Y \otimes X' \\
        \dTo^{g \ltimes X} &                        & \dTo_{g \ltimes X'} \\
        Y' \otimes X       & \rTo^{Y' \rtimes J_Af} & Y' \otimes X',
    \end{diagram}
    \]
    \end{column}
    \end{columns}
    \vspace{1em}
    need to commute.
\end{proof}
\end{frame}

\begin{frame}
\begin{proof}[Proof (contd.)]
\renewcommand{\qedsymbol}{}
    Following the definitions the first diagram translates to
    \[
    \begin{diagram}
        X \otimes Y         & \rTo^{first_Y(arr(f))}    & X' \otimes Y \\
        \dTo^{second_X(g)}  &                           & \dTo_{second_{X'}(g)} \\
        X \otimes Y'        & \rTo^{first_{Y'}(arr(f))} & X' \otimes Y'
    \end{diagram}
    \]
\end{proof}
\end{frame}

\begin{frame}
\begin{proof}[Proof (contd.)]
    \begin{align*}
        & first_Y(arr(f)) \ggg second_{X'}(g) \\
          & \quad = arr(f \times id_Y) \ggg second_{X'}(g) \\
          & \quad = arr(f \times id_Y) \ggg arr(\gamma_{X'Y}) \ggg first_{X'}(g) \ggg arr(\gamma_{Y'X'}) \\
          & \quad = arr((f \times id_Y) \circ \gamma_{X'Y}) \ggg first_{X'}(g) \ggg arr(\gamma_{Y'X'}) \\
          & \quad = arr(\gamma_{XY} \circ (id_Y \times f)) \ggg first_{X'}(g) \ggg arr(\gamma_{Y'X'}) \\
          & \quad = arr(\gamma_{XY}) \ggg arr(id_Y \times f) \ggg first_{X'}(g) \ggg arr(\gamma_{Y'X'}) \\
          & \quad = arr(\gamma_{XY}) \ggg first_{X}(g) \ggg arr(id_{Y'} \times f) \ggg arr(\gamma_{Y'X'}) \\
          & \quad = arr(\gamma_{XY}) \ggg first_{X}(g) \ggg arr((id_{Y'} \times f) \circ \gamma_{Y'X'}) \\
          & \quad = arr(\gamma_{XY}) \ggg first_{X}(g) \ggg arr(\gamma_{Y'X} \circ (f \times id_{Y'})) \\
          & \quad = arr(\gamma_{XY}) \ggg first_{X}(g) \ggg arr(\gamma_{Y'X}) \ggg arr(f \times id_{Y'}) \\
          & \quad = arr(\gamma_{XY}) \ggg first_{X}(g) \ggg arr(\gamma_{Y'X}) \ggg first_{Y'}(arr(f)) \\
          & \quad = second_X(g) \ggg first_{Y'}(arr(f)).
    \end{align*}
\end{proof}
\end{frame}

\begin{frame}
\begin{lemma}
    Every Freyd category of the form $\catD, \catC, J: \catC \to \catD$ induces
    an arrow.
\end{lemma}
\end{frame}

\begin{frame}
\begin{proof}
\renewcommand{\qedsymbol}{}
    To define $A: \opcat{\catC} \times \catC \to \catC$, set $A(X, Y) :=
    \catD(X,Y)$. \\
    \vspace{1em}
    We first need to verify that $A$ can be made a monoid in the category
    of $\catC$-enriched profunctors $\opcat{\catC} \times \catC \to \catC$.\\
    \vspace{1em}
    Set
    \begin{itemize}
        \item $arr(f) := Jf$,
        \item $a \ggg b := b \circ_\catD a$.
    \end{itemize}
    The arrow laws (1)--(4) hold as associativity ((1)) is inherited from
    $\catD$, and $J$ as a functor distributes ((2)) and preserves identities ((3),
    (4)).
\end{proof}
\end{frame}

\begin{frame}
\begin{proof}[Proof (contd.)]
$first$ can be given as
    \[
        (arr(\lambda p : X \times Z. (p, p))
          \ggg ((arr(\pi_1) \ggg a) \ltimes X \times Z))
          \ggg arr(id_Y \times \pi_2).
    \]
The checking of arrow laws (5)--(9) is omitted.
\end{proof}
\end{frame}

\begin{frame}
Combining the lemmas we get the
\begin{theorem}[``Arrows are Freyd categories'']
    Given any locally small category $\catC$ with finite products, there is a
    one-to-one correspondence between arrows $A$ over $\catC$ and locally small
    Freyd categories $\catC \to \catD$.
\end{theorem}
\end{frame}
